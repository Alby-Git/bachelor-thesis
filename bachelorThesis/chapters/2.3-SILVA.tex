\section{SILVA-database}\label{sec:SILVA}
SILVA is an ever-expanding online collection of databases containing quality-checked rRNA sequences~\cite{quast_silva_2013}. In February 2007, its initial release 89 contained 353,366 SSU sequences and 46,979 LSU sequences~\cite{noauthor_release_nodate-1}. The last release v138.1 (August 2020) comprises 9,469,124 SSU sequences and 1,312,534 LSU sequences~\cite{noauthor_release_nodate}. The database is periodically issued as releases, rather than undergoing continuous updates, with the aim of enabling comparability of studies making use of the SILVA database. The data is derived from the EMBL-Bank, each SILVA release is assigned a number corresponding to the EMBL-Bank release from which it originates~\cite{quast_silva_2013}. 
\\
The SILVA databases serve as the official databases for ARB, a software package  employed for database handling and data analysis. The databases are available for download via the SILVA website in either ARB or FASTA format files. This flexibility allows researchers to integrate SILVA with the usage of various pipelines/tools such as QIIME, Mothur etc.~\cite{quast_silva_2013}.
\\
SILVA offers three distinct database categories for both SSU and LSU sequences. Firstly, the Parc database encompasses all sequences available in SILVA. Second, the Ref database, which is a refined subset of the Parc database, containing only high-quality sequences. Lastly, the Ref NR 99 database, derived from the Ref database by excluding sequences that share 99\%\ or more similarity with one another~\cite{noauthor_release_nodate}.
\\
A Benchmark conducted by Almeida \emph{et al}. compared taxonomic classifiers (MAPseq v1.2.2, mothur v1.39.5, QIIME v1.9.1, and QIIME 2 v2017.11) in conjunction with diverse databases (Greengenes v13\textunderscore8, SILVA v128, RDP v16, and NCBI mapref v2.2) using simulated samples. The findings of this research revealed that SILVA generally demonstrated superior recall rates at the family and genus ranks. Moreover, SILVA had better accuracy in predicting the true composition of the mock samples~\cite{almeida_benchmarking_2018}.