\chapter{Introduction}\label{chap:Introduction}

Amplicon sequencing is a research method, designed to uncover the taxonomic composition of microbial communities. Ribosomal ribonucleic acid (rRNA) or ribosomal deoxyribonucleic acid (rDNA) are sequenced by targeting specific genes such as 16S for bacteria and archaea, 18S for eukaryota, or internal transcribed spacers (ITS) regions for fungi. Amplicon sequencing, compared to whole genome sequencing, is cost-effective and consumes less time~\cite{fricker_what_2019}. Amplicon sequencing is employed in a broad range of fields such as cancer research~\cite{guerrero-preston_16s_2016}, forensics~\cite{zhang_application_2023}, and food safety~\cite{mira_miralles_high-throughput_2019}.\\
Various pipelines/tools can be used in conjunction of different databases for amplicon analysis. Given the absence of a definitive singular approach, several studies were conducted (mentioned in \Cref{chap:State_of_the_art}) in recent years. These studies compared time cost, memory usage, and accuracy, across the spectrum of tools when paired with the different databases.

\section{MGnify}\label{sec:MGnify-intro}

MGnify offers a free-to-use service for the analysis and archiving of microbiome data.
The pipelines and used tools are open source (GitHub~\cite{noauthor_mgnify_nodate-22}). Each workflow is comprehensively documented using the Common Workflow Language (CWL)~\cite{language_cwl_home_nodate}, enhancing the reproducibility of the pipelines~\cite{mitchell_mgnify_2020}. Version 5.0 of MGnify offers three analysis pipelines associated with distinct input types: amplicon data, raw metagenomic reads, and assembly~\cite{noauthor_mgnify_nodate-25}. At the time of writing this thesis, MGnify had analyzed 356,039 amplicon analyses, 28,873 assemblies, and 33,827 metagenomic analyses~\cite{noauthor_mgnify_nodate-23}.
\section{Galaxy project}\label{sec:Galaxy-intro}
Galaxy is a free-to-use web-based platform for scientific computing. It offers a user-friendly interface for running tools/workflows on large data-sets, the workflow manager allows users to create new workflows, but also to import and edit already existing ones~\cite{the_galaxy_community_galaxy_2022}. Galaxy offers a wide range of tutorials covering various scientific and technical topics, by the time of writing 321 contributors have contributed their work on 360 tutorials covering 37 topics such as metagenomics, transcriptomics, variant analysis etc.~\cite{noauthor_galaxy_0000}.

\section{Aim of this thesis}\label{aim_of_thesis}
Although MGnify is documented using CWL, the pipeline is technically difficult to execute and requires vast computational resources. Therefore, individual users can only use the analysis service provided by MGnify. To facilitate individual execution of MGnify's analysis pipeline and concurrently offer computational resources, this thesis aims to investigate the feasibility to port the MGnify pipelines v5.0 into the Galaxy framework. The availability on Galaxy will offer several additional advantages such as interoperability, reproducibility, shareability, integration with MGnify Jupyter Notebooks, exchange of specific tools (e.g. for quality control), as well as downstream analysis using machine learning and differential abundance analysis .\\
Due to the limited time scope of this bachelor thesis, the efforts had to be concentrated on a single pipeline. The decision was made to port the amplicon pipeline. Given the commonality of the rRNA-prediction subworkflow across all three pipelines~\cite{mgnify_mgnify_2023}, it was selected as the component for integration. The availability of quality-checked reads in MGnify, which serve as an input to the rRNA-prediction subworkflow, offered the opportunity to bypass the quality control subworkflow and facilitate benchmarking of the ported version of the rRNA-prediction subworkflow against its original counterpart. The ported pipeline was benchmarked using beta diversity metrics to measure the dissimilarity values between the taxonomic abundance outputs for different taxonomic ranks.\\
Building upon prior studies (\Cref{sec:Kraken}) that found Kraken v2 to be superior in some cases as compared to purely amplicon based pipelines, the ported subworkflow was furthermore compared to Kraken v2.