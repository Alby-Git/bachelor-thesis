\chapter{Discussion and outlook}\label{chap:Discussion}
In this thesis, the rRNA-prediction subworkflow from the MGnify amplicon pipeline v5.0 was successfully ported to Galaxy. The process included integration or substitution of absent tools and reconstruction of the subworkflow on Galaxy, which resulted in an available, working subworkflow within the Galaxy platform. The link to Galaxy-ported rRNA-prediction subworkflow can be found in \Cref{appendix_galaxy_workflows}.\par
The beta diversity-based benchmark between MGnify and Galaxy-port, performed on samples from previous MGnify analyses, showed overall high similarity. The two beta diversity metrics, employed during this thesis, indicated low dissimilarity values for higher taxonomic ranks. The Jaccard distance values were consistently higher than the Bray Curtis distance values, which resulted from minority taxa (taxa with low abundance) being present in MGnify results and absent in Galaxy-port results, or vice versa.\\
The dissimilarities could be attributed to the inconsistency of the tool MAPseq. The taxonomic classifier, MAPseq, appeared to have a non-deterministic algorithm, which lead to slightly inconsistent outputs. MAPseq, when tested multiple times using identical command, configuration, input, and database files, occasionally delivered slightly different outputs. This issue of MAPseq was brought to the attention of both the MGnify team and the MAPseq developers, seeking clarification.\par
The beta diversity benchmark between MGnify and Kraken v2, conducted on samples from previous MGnify analyses, delivered notably high dissimilarity values. Several factors might have contributed to the high dissimilarity. Firstly, Kraken v2 used a 16S SILVA database, while the MGnify samples originate from SSU analyses. Since 16S and 18S SSU analyses are not distinctly categorized in MGnify, it cannot not be ruled out that the MGnify samples include 18S SSUs as well. Secondly, Kraken v2 used a newer SILVA database version (v138) compared to MGnify (SILVA v132). The choice to use Kraken v2 in conjunction with SILVA 16S v138 was driven by its availability on the Galaxy platform, as SILVA 16S v132 was not accessible. Lastly, the SILVA v138 database files, compatible for use with MGnify are currently unavailable. According to SILVA these files are unlikely to be available before 2025.\par
A more informative comparison between Kraken v2 and the rRNA-prediction subworkflow, was conducted on 16S mock samples. The benchmark compared the results of the Galaxy-ported rRNA-prediction and Kraken v2 against the expected composition of the mock samples. The choice of 16S mock samples provided a more robust comparison, as Kraken v2 used the 16S SILVA database. Despite Kraken v2 using a newer SILVA database version (v138) in comparison to the Galaxy-ported version (SILVA v132), the rRNA-prediction subworkflow on Galaxy consistently outperformed Kraken v2. The Galaxy-ported rRNA-prediction subworkflow yielded lower dissimilarity values than Kraken v2, indicating that it was closer to the expected composition of the mock samples. This observation is additionally supported by the generated relative abundance plot for the genus rank, which indicated higher abundances of false-positive taxa for Kraken v2 in comparison to the Galaxy-ported rRNA-prediction subworkfow.\par
In summary, the integration and construction of the rRNA-prediction subworkflow, part of MGnify amplicon pipeline v5.0, within Galaxy was executed successfully. The Benchmark, comparing the Galaxy-ported version against its original counterpart, demonstrated good agreement between results. However, slight differences were detected, towards lower taxonomic ranks. Regarding both MGnify and Galaxy-ported results, the super kingdom and kingdom ranks exhibited identical taxa presence and nearly identical taxa abundance.\\
While the recent study by Odom \emph{et al.} reported a strong performance of Kraken v2~\cite{odom_metagenomic_2023} for amplicon prediction, the rRNA-prediction subworkflow consistently outperformed Kraken v2.\\
The availability of the workflow on Galaxy facilitates interoperability, shareability, integration with MGnify Jupyter Notebooks, downstream analysis using machine learning and differential abundance analysis. Additionally, Galaxy facilitates exchange and modification of specific tools. However, results reproducibility and comparability are limited due to the non-deterministic algorithm of MAPseq.\par
In future work, the Galaxy-ported rRNA-prediction subworkflow can be further evaluated using samples from LSU MGnify analyses. Since rRNA-prediction is a common subworkflow across all three MGnify pipelines amplicon, raw metagenomic reads, and assembly, its availability on Galaxy supports future work on these pipelines. Ongoing work is dedicated to port the quality control and ITS subworkflows, of the amplicon pipeline, which will soon make the entire pipeline available on Galaxy.