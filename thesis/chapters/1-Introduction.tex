\chapter{Introduction}\label{chap:Introduction}

Metagenomics is an expansive field focused on the analysis of genetic material recovered directly from environmental samples. 
This approach provides a comprehensive view of the microbial communities present in a given habitat.
A critical component of metagenomic analysis involves taxonomic profiling, which identifies and categorizes the various microorganisms within a sample. 
This profiling is essential for understanding microbial diversity, ecosystem functioning, and the roles of specific microbes in health and disease.

Various tools and pipelines have been developed to facilitate taxonomic profiling, each leveraging different databases and algorithms.
Among these tools, Kraken2 and Bracken have gained prominence for their high accuracy and efficiency in classifying reads from metagenomic samples.
Kraken2 employs a k-mer based approach to classify reads, while Bracken refines these classifications to improve abundance estimates. 
Similarly, MetaPhlAn and mOTUs are widely used for profiling microbial communities using marker genes.

Despite the availability of these tools, there is no consensus on a single best approach for metagenomic analysis. 
Comparative studies are necessary to evaluate their performance in terms of time efficiency, memory usage, and accuracy. 
These studies help in identifying the strengths and limitations of each tool, providing valuable insights for selecting the appropriate pipeline for specific research needs.


\section{CAMI Challenge}\label{sec:CAMI-challange}
The Critical Assessment of Metagenome Interpretation (CAMI) challenge provides a rigorous benchmarking platform for metagenomic tools and pipelines.
CAMI was initiated to address the need for standardized evaluation in the rapidly evolving field of metagenomics.
It aims to provide a common ground for developers and users to assess the performance of different methods under controlled conditions.

By offering a gold standard dataset, CAMI allows researchers to evaluate the performance of different methods in a controlled setting. 
The CAMI datasets are meticulously curated and designed to reflect realistic microbial community compositions, 
making them highly relevant for benchmarking purposes. The challenge encompasses various tasks such as taxonomic profiling, genome assembly,
and binning, providing a comprehensive evaluation framework.

The importance of the CAMI challenge in my thesis cannot be overstated. It provides a robust,
unbiased benchmark against which the performance of Kraken2, Bracken, MetaPhlAn, and mOTUs can be measured. 
This benchmark is crucial for validating the accuracy and reliability of these tools when applied to real-world metagenomic data. 
The insights gained from this benchmarking study will not only help in identifying the most effective tool 
for taxonomic profiling but also guide future improvements in metagenomic analysis pipelines.
\newpage
\section{Galaxy project}\label{sec:Galaxy-intro}
Galaxy is a versatile, web-based platform designed for accessible and reproducible scientific computing. 
It supports a wide array of bioinformatics tools and workflows, enabling researchers to process large datasets efficiently. 
The platform's user-friendly interface and comprehensive workflow management capabilities make it an ideal environment for developing and sharing complex analytical pipelines.

One of the key strengths of Galaxy is its ability to integrate tools and datasets seamlessly. 
Users can create, import, and modify workflows with ease, facilitating a high degree of customization and flexibility.
Galaxy also supports the reproducibility of scientific analyses by maintaining detailed records of the workflows and parameters used in each experiment.

Galaxy's community-driven development ensures that it stays current with the latest advancements in bioinformatics. 
The platform offers extensive documentation and a wealth of tutorials, covering a wide range of scientific and technical topics.
This makes it accessible to researchers with varying levels of expertise and enhances its utility across diverse research applications.

\newpage
\section{Aim of this thesis}\label{aim_of_thesis}
The primary aim of this thesis is to benchmark the performance of Kraken2, Bracken, MetaPhlAn, and mOTUs using the CAMI challenge dataset as a gold standard. 
The analysis will be conducted within the Galaxy framework to leverage its robust computational resources and workflow management capabilities. 
This study will compare the taxonomic profiling outputs of these tools against the gold standard provided by CAMI, using Cami Opal for detailed comparative analysis.
By integrating these tools into Galaxy and conducting a thorough benchmarking study, 
this thesis seeks to provide insights into their relative performance and suitability for different metagenomic research scenarios. Specifically, the thesis will:
\begin{itemize}
    \item Evaluate the accuracy of Kraken2, Bracken, MetaPhlAn, and mOTUs in taxonomic profiling against the CAMI gold standard dataset.
    \item Assess the computational efficiency and memory usage of these tools within the Galaxy framework.
    \item Identify the strengths and weaknesses of each tool, providing recommendations for their use in various metagenomic analysis contexts.
    \item Contribute to the refinement and improvement of metagenomic analysis pipelines by highlighting areas for future development.
\end{itemize}
The outcomes of this research will not only highlight the strengths and weaknesses of
each tool but also contribute to the ongoing efforts to refine metagenomic analysis pipelines, 
ultimately enhancing the accuracy and reliability of microbial community studies.