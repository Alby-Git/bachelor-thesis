\chapter*{Abstract}

Metagenomic sequencing is a powerful method used to uncover and characterize microbial communities by analyzing genetic material from environmental samples. This thesis benchmarks the performance of Kraken2, Bracken, MetaPhlAn, and mOTUs using the CAMI challenge dataset as a gold standard within the Galaxy framework.
The benchmarking process compared the taxonomic profiling outputs of these tools against the CAMI gold standard using Cami Opal for detailed analysis. Metrics such as taxonomic profiling accuracy, computational efficiency, and memory usage were assessed. Kraken2 and Bracken provided high accuracy, while MetaPhlAn and mOTUs effectively profiled microbial communities using marker genes.
Integrating these tools into Galaxy enhances their utility for metagenomic research by providing a reproducible and customizable analysis environment. The study highlights the strengths and limitations of each tool, offering valuable insights for selecting the appropriate pipeline for specific research needs.
The availability of these tools within Galaxy offers several advantages for the metagenomics research community, including interoperability, reproducibility, and ease of use, facilitating advanced downstream analyses such as machine learning and differential abundance analysis.

\vspace{1cm}

\textbf{Deutsche Version:}

\vspace{0.5cm}

Metagenomisches Sequenzieren ist eine leistungsstarke Methode zur Entdeckung und Charakterisierung mikrobieller Gemeinschaften durch die Analyse genetischen Materials aus Umweltproben. Diese Arbeit vergleicht die Leistung von Kraken2, Bracken, MetaPhlAn und mOTUs anhand des CAMI-Challenge-Datensatzes als Goldstandard im Galaxy-Framework.
Der Benchmarking-Prozess verglich die taxonomischen Profilierungsausgaben dieser Werkzeuge mit dem CAMI-Goldstandard unter Verwendung von Cami Opal für eine detaillierte Analyse. Metriken wie die Genauigkeit der taxonomischen Profilierung, die Recheneffizienz und die Speichernutzung wurden bewertet. Kraken2 und Bracken boten eine hohe Genauigkeit, während MetaPhlAn und mOTUs effektiv mikrobiellen Gemeinschaften mit Markergenen profilierten.
Die Integration dieser Werkzeuge in Galaxy erhöht deren Nutzen für die metagenomische Forschung durch eine reproduzierbare und anpassbare Analyseumgebung. Die Studie hebt die Stärken und Schwächen jedes Werkzeugs hervor und bietet wertvolle Einblicke in die Auswahl der geeigneten Pipeline für spezifische Forschungsanforderungen.
Die Verfügbarkeit dieser Werkzeuge im Galaxy-Framework bietet der Metagenomik-Forschungsgemeinschaft mehrere Vorteile, darunter Interoperabilität, Reproduzierbarkeit und Benutzerfreundlichkeit, und erleichtert fortschrittliche nachgelagerte Analysen wie maschinelles Lernen und Differenzanalyse der Abundanz.