\chapter*{Abstract}
Amplicon sequencing is a powerful taxonomic classification method used to discover and characterize microbial communities. In this thesis, the rRNA-prediction subworkflow, a part of the MGnify amplicon pipeline v5.0, was successfully ported to the Galaxy platform. Subsequently, the Galaxy-ported subworkflow was benchmarked against Kraken v2 and its MGnify counterpart using beta diversity and relative taxonomic abundance as benchmark measures. The benchmarking process involved samples from previous MGnify analyses, as well as mock samples.\\
The Galaxy-ported subworkflow consistently outperformed Kraken v2 for both sample types. The ported subworkflow produces overall similar results to MGnify, a slight discrepancy could be attributed to MAPseq. MAPseq, a tool within the rRNA-prediction subworkflow, appeared to be non-deterministic, which lead to slight differences between the results of MGnify and its Galaxy-ported version.\\
The availability of the rRNA-prediction subworkflow on Galaxy offers several advantages for the microbiome research community such as interoperability, exchange and modification of specific tools, and shareability, downstream applications such as machine learning and differential abundance analysis.


\emph{\textbf{Deutsche Version:}}\\
Die Amplicon-Sequenzierung ist eine leistungsstarke taxonomische Klassifikationsmethode, die zur Entdeckung von mikrobiellen Gemeinschaften verwendet wird. In dieser Arbeit wurde der rRNA-Prediction-Subworkflow, ein Teil des MGnify Amplicon-Pipeline v5.0, erfolgreich auf die Galaxy-Plattform portiert. Anschließend wurde der auf Galaxy portierte Subworkflow anhand von Beta-Diversität und relativer taxonomischer Häufigkeit mit Kraken v2 und dem MGnify-Subworkflow verglichen. Der Benchmarking-Prozess umfasste Proben aus früheren MGnify-Analysen sowie Mock-Proben.\\
Der auf Galaxy portierte Subworkflow übertraf Kraken v2 konsistent für beide Probentypen. Der portierte Subworkflow liefert insgesamt ähnliche Ergebnisse wie die MGnify-Ergebnisse, eine leichte Abweichung könnte auf MAPseq zurückzuführen sein. MAPseq, ein Tool innerhalb des rRNA-Prediction-Subworkflow, schien nicht deterministisch zu sein, was zu leichten Unterschieden zwischen den Ergebnissen von MGnify und seiner auf Galaxy portierten Version führte.\\
Die Verfügbarkeit des rRNA-Prediction-Subworkflows auf Galaxy bietet mehrere Vorteile, wie beispielsweise Interoperabilität, den Austausch und die Anpassung von spezifischen Tools sowie die Möglichkeit zu teilen.