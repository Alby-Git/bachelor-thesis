\section{Kraken v2}\label{sec:Kraken}
Kraken v2 is a k-mer-based taxonomic classification tool. It is the improved version of Kraken v1, it consumes 85\%\ less memory, is five times faster than Kraken v1, and still maintains high accuracy~\cite{wood_improved_2019}. These improvements are achieved by replacing the previously employed sorted list of k-mer/LCA (Lowest Common Ancestor) pairs, which was indexed by minimizers, with a more efficient probabilistic and compact hash table that maps minimizers to their respective LCAs~\cite{wood_improved_2019}. In contrast to Kraken v1, storing all k-mers, the data structure of Kraken v2 stores only the minimizers corresponding to every k-mer~\cite{wood_improved_2019}.
\\
The default taxonomy database used in combination with Kraken v2 is the NCBI database, it also supports three non-NCBI taxonomy based databases, such as Greengenes 16S sequences database, RDP 16S database, and SILVA SSU Ref NR 99. Additionally, a custom database can be build using known taxonomies~\cite{noauthor_kraken2docsmanualmarkdown_nodate}.
\\
In a study conducted by Lu and Salzberg in 2020, a comparison was made among Kraken v2, Bracken v2.5, and QIIME 2 v2017.11, all used in conjunction with the RDP 11.5, SILVA 132, and Greengenes v13\textunderscore8 databases. These tools were evaluated using the same simulated data-sets from a study by Almeida \emph{et al.}~\cite{almeida_benchmarking_2018}. The findings of the study revealed that Kraken v2 and Bracken significantly outperformed QIIME~2 in terms of database build time for the Greengenes and SILVA databases, being up to 100 times faster. Furthermore, in terms of classification time, Kraken v2 and Bracken were up to 300 times faster, consumed up to 100 times less RAM, and demonstrated higher 16S rRNA profiling accuracy compared to QIIME~2~\cite{lu_ultrafast_2020}.\\
Another recent study by Odom \emph{et al.} compared Kraken v2, Mothur, PathoScope~2, QIIME~2, and DADA~2, in conjunction with Greengenes v13\textunderscore8, Kraken v2, SILVA v138, and Refseq v2020 reference databases. The study was conducted using 16S simulated samples retrieved from multiple sources. The study's results showed that the whole-genome metagenomics tools Kraken v2 and PathoScope~2, demonstrated better performance than the 16S analyses tools DADA~2, QIIME~2, and Mothur~\cite{odom_metagenomic_2023}.